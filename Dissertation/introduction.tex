\chapter*{Введение}                         % Заголовок
\addcontentsline{toc}{chapter}{Введение}    % Добавляем его в оглавление

\newcommand{\actuality}{}
\newcommand{\progress}{}
\newcommand{\aim}{{\textbf\aimTXT}}
\newcommand{\tasks}{\textbf{\tasksTXT}}
\newcommand{\novelty}{\textbf{\noveltyTXT}}
\newcommand{\influence}{\textbf{\influenceTXT}}
\newcommand{\methods}{\textbf{\methodsTXT}}
\newcommand{\defpositions}{\textbf{\defpositionsTXT}}
\newcommand{\reliability}{\textbf{\reliabilityTXT}}
\newcommand{\probation}{\textbf{\probationTXT}}
\newcommand{\contribution}{\textbf{\contributionTXT}}
\newcommand{\publications}{\textbf{\publicationsTXT}}

{\actuality} 
\textbf{Актуальность темы.}
Количественный и качественный анализ физических особенностей потоков тепла и их изменчивости --  одна из основных проблем современных океанологии и климатологии. Существует целый ряд открытых вопросов и проблем при количественной оценке тепловых потоков в процессе взаимодействия океан--атмосфера:
\begin{itemize}
	\item Положение и изменения областей с максимальным взаимодействием;
	\item Области преобладания оттока в атмосферу, перераспределения внутри океана;
	\item Сезонные и межгодовые тренды и колебания. 
\end{itemize}

\noindent
\textbf{Мотивация выбора математической модели}
\begin{itemize}
	\item Для понимания и количественного описания взаимодействия океана и атмосферы существуют различные модели и методы, например, основанные на уравнениях гидро-термодинамики. 
	\item Модель СДУ Ито достаточно проста, поскольку математически задача сводится к определению двух коэффициентов (вектора сноса и матрицы диффузии), зная которые, можно с помощью хорошо разработанного математического аппарата получить количественные оценки поведения изучаемых характеристик, провести их анализ и сделать прогноз с оценкой надежности.
	\item С другой стороны, эта модель достаточно общая, так как она включает в себя как динамические модели со случайным форсингом, так и схемы с разделением процесса на тренды, периодическую и случайную составляющую.
\end{itemize}

%\ifsynopsis
%Этот абзац появляется только в~автореферате.

%\else
%Этот абзац появляется только в~диссертации.
%\fi

% {\progress}
% Этот раздел должен быть отдельным структурным элементом по
% ГОСТ, но он, как правило, включается в описание актуальности
% темы. Нужен он отдельным структурынм элемементом или нет ---
% смотрите другие диссертации вашего совета, скорее всего не нужен.

{\aim} данной работы является стохастическое описание динамики процессов и их взаимодействия с помощью математической модели типа стохастического дифференциального уравнения (СДУ) Ито со случайными коэффициентами сдвига и диффузии, а также использование метода Карунена--Лоэва разложения на ортогональные компоненты.

Для~достижения поставленной цели необходимо было решить следующие {\tasks}:
\begin{enumerate}[beginpenalty=10000] % https://tex.stackexchange.com/a/476052/104425
  \item Исследовать, разработать, вычислить и~т.\:д. и~т.\:п.
  %\item Исследовать, разработать, вычислить и~т.\:д. и~т.\:п.
  %\item Исследовать, разработать, вычислить и~т.\:д. и~т.\:п.
  %\item Исследовать, разработать, вычислить и~т.\:д. и~т.\:п.
\end{enumerate}


{\novelty}
Все результаты, полученные в данной научно-квалификационной
работе, являются новыми.
%\begin{enumerate}[beginpenalty=10000] % https://tex.stackexchange.com/a/476052/104425
	
%\item Впервые \ldots
%  \item Впервые \ldots
%  \item Было выполнено оригинальное исследование \ldots
%\end{enumerate}

{\influence} данной работы заключается в ... TODO

{\methods} В научно-квалификационной работе используются методы математического моделирования, теории вероятностей, математической статистики.

{\defpositions}
\begin{enumerate}[beginpenalty=10000] % https://tex.stackexchange.com/a/476052/104425
  \item TODO
  \item TODO Сравнение методов реконструкции оценок на модельных и реальных данных
  \item TODO Реализация разложения оценок коэффициента диффузии на собственные вектора на основе алгоритма Карунена-Лоэва
  %\item Четвертое положение
\end{enumerate}
%В папке Documents можно ознакомиться с решением совета из Томского~ГУ
%(в~файле \verb+Def_positions.pdf+), где обоснованно даются рекомендации
%по~формулировкам защищаемых положений.

{\reliability} полученных результатов обеспечивается \ldots \ Результаты находятся в соответствии с результатами, полученными другими авторами.


{\probation}
Основные результаты работы докладывались автором
на следующих конференциях:
\begin{enumerate}[beginpenalty=10000]
	\item Осипова А. А., Горшенин А. К., Беляев К. П. Стохастические методы совместного анализа данных температуры воды, потоков тепла и атмосферного давления в Северной Атлантике // Тезисы докладов научной конференции Тихоновские чтения (2024 г., МАКС Пресс, Москва, тезисы). — Т. 46 из Издательский отдел факультета ВМК МГУ имени М.В. Ломоносова. — Москва: ООО МАКС Пресс, 2024. — С. 115.
	\item Осипова А. А., Горшенин А. К. О статистическом оценивании параметров динамико-стохастической модели потоков тепла между океаном и атмосферой // Ломоносовские чтения-2022: научная конференция, факультет ВМК МГУ имени М.В.Ломоносова. Тезисы докладов. — Т. 2022 из СЕКЦИЯ ВЫЧИСЛИТЕЛЬНОЙ МАТЕМАТИКИ И КИБЕРНЕТИКИ. — Москва: ООО МАКС Пресс, 2022. — С. 170–171.
	\item Осипова А. А., Горшенин А. К. Прогнозирование временных рядов на основе гребневой регрессии и расширения признакового пространства // Тихоновские чтения: научная конференция: 25–30 октября 2021 г. : тезисы докладов. — Т. 46. — Москва: ООО МАКС Пресс, 2021. — С. 124–124.
	\item Осипова А. А., Горшенин А. К., Беляев К. П. Совместный анализ температуры воды,
	потоков тепла и атмосферного давления в
	Северной Атлантике // Доклад. Институт океанологии им. П.П. Ширшова Российской академии наук. 27 сентября 2024 года
\end{enumerate}

%{\contribution} Автор принимал активное участие \ldots

\ifnumequal{\value{bibliosel}}{0}
{%%% Встроенная реализация с загрузкой файла через движок bibtex8. (При желании, внутри можно использовать обычные ссылки, наподобие `\cite{vakbib1,vakbib2}`).
    {\publications} Основные результаты по теме диссертации изложены
    в~XX~печатных изданиях,
    X из которых изданы в журналах, рекомендованных ВАК,
    X "--- в тезисах докладов.
}%
{%%% Реализация пакетом biblatex через движок biber
    \begin{refsection}[bl-author, bl-registered]
        % Это refsection=1.
        % Процитированные здесь работы:
        %  * подсчитываются, для автоматического составления фразы "Основные результаты ..."
        %  * попадают в авторскую библиографию, при usefootcite==0 и стиле `\insertbiblioauthor` или `\insertbiblioauthorgrouped`
        %  * нумеруются там в зависимости от порядка команд `\printbibliography` в этом разделе.
        %  * при использовании `\insertbiblioauthorgrouped`, порядок команд `\printbibliography` в нём должен быть тем же (см. biblio/biblatex.tex)
        %
        % Невидимый библиографический список для подсчёта количества публикаций:
        \phantom{\printbibliography[heading=nobibheading, section=1, env=countauthorvak,          keyword=biblioauthorvak]%
        \printbibliography[heading=nobibheading, section=1, env=countauthorwos,          keyword=biblioauthorwos]%
        \printbibliography[heading=nobibheading, section=1, env=countauthorscopus,       keyword=biblioauthorscopus]%
        \printbibliography[heading=nobibheading, section=1, env=countauthorconf,         keyword=biblioauthorconf]%
        \printbibliography[heading=nobibheading, section=1, env=countauthorother,        keyword=biblioauthorother]%
        \printbibliography[heading=nobibheading, section=1, env=countregistered,         keyword=biblioregistered]%
        \printbibliography[heading=nobibheading, section=1, env=countauthorpatent,       keyword=biblioauthorpatent]%
        \printbibliography[heading=nobibheading, section=1, env=countauthorprogram,      keyword=biblioauthorprogram]%
        \printbibliography[heading=nobibheading, section=1, env=countauthor,             keyword=biblioauthor]%
        \printbibliography[heading=nobibheading, section=1, env=countauthorvakscopuswos, filter=vakscopuswos]%
        \printbibliography[heading=nobibheading, section=1, env=countauthorscopuswos,    filter=scopuswos]}%
        %
        \nocite{*}%
        %
        {\publications} Основные результаты по теме диссертации изложены в~\arabic{citeauthor}~печатных изданиях,
        \arabic{citeauthorvak} из которых изданы в журналах, рекомендованных ВАК%
        \ifnum \value{citeauthorscopuswos}>0%
            , \arabic{citeauthorscopuswos} "--- в~периодических научных журналах, индексируемых Web of~Science и Scopus%
        \fi%
        \ifnum \value{citeauthorconf}>0%
            , \arabic{citeauthorconf} "--- в~тезисах докладов.
        \else%
            .
        \fi%
        \ifnum \value{citeregistered}=1%
            \ifnum \value{citeauthorpatent}=1%
                Зарегистрирован \arabic{citeauthorpatent} патент.
            \fi%
            \ifnum \value{citeauthorprogram}=1%
                Зарегистрирована \arabic{citeauthorprogram} программа для ЭВМ.
            \fi%
        \fi%
        \ifnum \value{citeregistered}>1%
            Зарегистрированы\ %
            \ifnum \value{citeauthorpatent}>0%
            \formbytotal{citeauthorpatent}{патент}{}{а}{}%
            \ifnum \value{citeauthorprogram}=0 . \else \ и~\fi%
            \fi%
            \ifnum \value{citeauthorprogram}>0%
            \formbytotal{citeauthorprogram}{программ}{а}{ы}{} для ЭВМ.
            \fi%
        \fi%
        % К публикациям, в которых излагаются основные научные результаты диссертации на соискание учёной
        % степени, в рецензируемых изданиях приравниваются патенты на изобретения, патенты (свидетельства) на
        % полезную модель, патенты на промышленный образец, патенты на селекционные достижения, свидетельства
        % на программу для электронных вычислительных машин, базу данных, топологию интегральных микросхем,
        % зарегистрированные в установленном порядке.(в ред. Постановления Правительства РФ от 21.04.2016 N 335)
    \end{refsection}%
    \begin{refsection}[bl-author, bl-registered]
        % Это refsection=2.
        % Процитированные здесь работы:
        %  * попадают в авторскую библиографию, при usefootcite==0 и стиле `\insertbiblioauthorimportant`.
        %  * ни на что не влияют в противном случае
        \nocite{vakbib2}%vak
        \nocite{patbib1}%patent
        \nocite{progbib1}%program
        \nocite{bib1}%other
        \nocite{confbib1}%conf
    \end{refsection}%
        %
        % Всё, что вне этих двух refsection, это refsection=0,
        %  * для диссертации - это нормальные ссылки, попадающие в обычную библиографию
        %  * для автореферата:
        %     * при usefootcite==0, ссылка корректно сработает только для источника из `external.bib`. Для своих работ --- напечатает "[0]" (и даже Warning не вылезет).
        %     * при usefootcite==1, ссылка сработает нормально. В авторской библиографии будут только процитированные в refsection=0 работы.
}
 % Характеристика работы по структуре во введении и в автореферате не отличается (ГОСТ Р 7.0.11, пункты 5.3.1 и 9.2.1), потому её загружаем из одного и того же внешнего файла, предварительно задав форму выделения некоторым параметрам

\textbf{Объем и структура работы.} Диссертация состоит из~введения,
\formbytotal{totalchapter}{глав}{ы}{}{},
заключения и
\formbytotal{totalappendix}{приложен}{ия}{ий}{}.
%% на случай ошибок оставляю исходный кусок на месте, закомментированным
%Полный объём диссертации составляет  \ref*{TotPages}~страницу
%с~\totalfigures{}~рисунками и~\totaltables{}~таблицами. Список литературы
%содержит \total{citenum}~наименований.
%
Полный объём диссертации составляет
\formbytotal{TotPages}{страниц}{у}{ы}{}, включая
\formbytotal{totalcount@figure}{рисун}{ок}{ка}{ков} и
\formbytotal{totalcount@table}{таблиц}{у}{ы}{}.
Список литературы содержит
\formbytotal{citenum}{наименован}{ие}{ия}{ий}.


%Диссертация организована следующим образом:
% В разделе 2 приведен обзор известных подходов к описанию поведения явного и скрытого потоков тепла, применявшихся за последние 10 лет.
%раздел~\ref{SecModel} содержит описание математической модели, применяющейся в данной работе, а также информацию об анализируемых данных. Раздел~\ref{SecMethods} посвящен вычислительным методам оценивания случайных коэффициентов СДУ Ланжевена. В разделе~\ref{SecSoftware} обсуждается архитектура программного комплекса, реализованного для проведения стохастического анализа потоков тепла с использованием высокопроизводительных вычислительных ресурсов. В разделе~\ref{SecAnalysis} проводится анализ поведения различных характеристик полученных оценок (максимум, минимум, среднее), а также исследуется взаимосвязь между оценками и значениями потоков, на основе которых они были построены. В разделе~\ref{SecDiscuss} обсуждаются полученые результаты и их связь с ранее установленными в данной области эффектами. В Приложении~\ref{AppExist} продемонстирована корректность применения стохастической модели для анализируемых данных. 
