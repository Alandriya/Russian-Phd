\appendix
\chapter{Проверка существования решения СДУ на используемых данных}
\label{app:Existance}
Решением СДУ вида~\eqref{eq:Ito} является диффузионный процесс с коэффициентом диффузии $b^2(t,X)$ и коэффициентом переноса $a(t,X)$. Случайные коэффициенты $a(t,X)$ и $b(t,X)$ представляют собой условное математическое ожидание и дисперсию приращений потока соответственно:
\begin{gather*}
	a(t,X) = \frac{\E(X(t+dt)-X(t)|X(t)=x)}{dt}, \quad
	% 		\end{equation}
% 		\begin{equation}
	b(t,X) = \frac{\D(X(t+dt)-X(t)|X(t)=x)}{dt}.
\end{gather*}

Предположим, что $a(t,X)$ и $b(t,X)$ –- борелевские функции, определенные при $x \in \R^1$, $t \in [t_0,T]$. Тогда рассматриваемое СДУ эквивалентно следующему уравнению (начальное условие $X(t_0)$ предполагается заданным):
\begin{equation}
	\label{integral_Ito}
	X(t) = X(t_0) + \int_{t_0}^{t} a(s, X(s))ds + \int_{t_0}^{t} b(s, X(s))dW(s).
\end{equation}

Для существования решения уравнения~\eqref{integral_Ito} для некоторого $K$ необходимо~\cite{Skorohod} выполнение условий вида:
\begin{enumerate}
	\item для всех $x$ и $y \in \R^1$: 
	% 			\begin{equation*}
		$|a(t,x)-a(t,y)|+|b(t,x)-b(t,y)| \leqslant K|x-y|$,
		% 			\end{equation*}
	\item для всех $x \in \R^1$: 
	% 			\begin{equation*}\
		$|a(t,x)|^2+|b(t,x)|^2 \leqslant K(1+x^2)$.
		% 			\end{equation*}
\end{enumerate}

Причем если при этом $X_1 (t)$ и $X_2(t)$ -– два непрерывных решения уравнения \ref{integral_Ito}, то они неразличимы.
% 		\begin{equation*}
	% 			\P \left\lbrace \sup_{t_0 \leqslant t \leqslant T} |X_1(t) - X_2(t)| > 0 \right\rbrace = 0.
	% 		\end{equation*}
% 		Более того, при условиях этой теоремы решение уравнения~\ref{integral_Ito} будет процессом Маркова, вероятности перехода которого определяются соотношением:
% 		\begin{equation*}
	% 			P(t,x,s,A)= P\left\lbrace X_(t,x) (s) \in A\right\rbrace 
	% 		\end{equation*}

Проверим выполнение этих условий для анализируемых данных, а именно -- найдем оценку константы $K$. Для неравенства из первого пункта естественным образом будем рассматривать случай $x \ne y$. Имеем:
% 		Оценим константу $K$ в этом неравенстве отдельно для каждого потока следующим образом: найдем минимум выражения $|x-y|$ при $x \ne y$ для всех значений потоков $x$ и $y$, имеющихся в данных. Затем для каждого $t \in [t_0,T]$ максимизируем разности $|a(t,x)-a(t,y)|+|b(t,x)-b(t,y)|$ и возьмем максимум по $t$. Тогда для 
\begin{equation*}
	K \geqslant \frac{\max(|a(t,x)-a(t,y)|+|b(t,x)-b(t,y)|)}{\min(|x-y|)}.
\end{equation*}
% 	первое неравенство будет выполнено для рассматриваемого типа потока. Для того, чтобы получить универсальную константу для обоих типов потоков, возьмем максимум из двух оценок.

Для второго неравенства, учитывая, что $(1+x^2 )\geqslant 1$, получим:
\begin{equation*}
	K \geqslant \max\limits_{x \in \R^1, t \in [t_0,T]}(|a(t,x)|^2 + |b(t,x)|^2).
\end{equation*}
% 	где максимум берется по всем присутствующих в данных значениям потока $x$ и при всех $t \in [t_0,T]$.

В данных присутствует небольшое количество выбросов, которые могут в несколько раз превосходить типичные значения коэффициентов. Для определения коэффцициентов в неравенствах они могут быть исключены: в качестве максимума для каждого типа потока выбирается квантиль порядка $0.97$, а в качестве минимума – $0.03$. 
В силу объема данных вычисление квантилей на всем промежутке времени одновременно затруднительно, поэтому квантили считались на отрезках по $1000$ дней, а затем в качестве верхней квантили брался максимум из них на каждом отрезке, а в качестве нижней –- минимум. Кроме того, введем условие отделимости от $0$ разности $x-y$, эмпирически выбран порог $0.1$.

С учетом введенных ограничений были получены следующие значения верхних и нижних квантилей для $a$ и $b$: $b_{up} = 148.02$, $b_{low} = 0$ (из физического смысла, случаи комплексных $b$ и потому отрицательных значений не рассматриваются), $a_{up} = 40.77$, $a_{low} = -32.02$. При таких введенных ограничениях получаем оценку на K для первого неравенства	$K_1 \geqslant 2209$ и для второго -- 	$K_2 \geqslant 23572$. 	Таким образом, необходимое конечное значение $K$ существует, гарантируя корректность применяемой в работе математической модели.


