%% Согласно ГОСТ Р 7.0.11-2011:
%% 5.3.3 В заключении диссертации излагают итоги выполненного исследования, рекомендации, перспективы дальнейшей разработки темы.
%% 9.2.3 В заключении автореферата диссертации излагают итоги данного исследования, рекомендации и перспективы дальнейшей разработки темы.
\begin{enumerate}
  \item На основе анализа \ldots
  \item Численные исследования показали, что \ldots
  \item Математическое моделирование показало \ldots
  \item Для выполнения поставленных задач был создан \ldots
\end{enumerate}

В работе описан метод статистического оценивания распределений случайных параметров стохастических дифференциальных уравнений типа Ито с помощью техники скользящего разделения смесей. Предложены дискретные аппроксимации для оценок указанных распределений. С целью изучения изменчивости распределений коэффициентов СДУ во времени предложен алгоритм последовательной идентификации (определения локальной связности) компонент получаемых смесей. В его основу положена комбинация жадного алгоритма для поиска числа компонент и одного из методов кластеризации ($k$- или $c$-средних). Функциональные параметры (компоненты распределения~\eqref{eq:DiffDiscrApprox} как функции времени), полученные в результате описываемых статистических процедур, могут быть использованы при обучении интеллектуальных алгоритмов прогнозирования процессов, удовлетворяющих уравнениям типа~\eqref{eq:Ito}. Применение метода иллюстрируется конкретными примерами анализа процесса теплообмена между атмосферой и океаном.